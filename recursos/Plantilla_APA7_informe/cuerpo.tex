\section{\large Título I}
% \noindent \maskCitet{cervantes1999}\\
En un lugar de la Mancha, de cuyo nombre no quiero acordarme, no ha mucho tiempo que vivía un hidalgo de los de lanza en astillero, adarga antigua, rocín flaco y galgo corredor.
\subsection{Título II}
Una olla de algo más vaca que carnero, salpicón las más noches, duelos y quebrantos los sábados, lantejas los viernes, algún palomino de añadidura los domingos, consumían las tres partes de su hacienda.
\begin{table}
    \caption{Sample words from this hypothetical experiment.}
    \centering
    \begin{tabular}{cc} %The c's here indicate the columns will be centered
        \hline
        First word & Second word \\
        \hline
        Yeet       & Yoink       \\
        Hot        & Lit         \\
        \hline
    \end{tabular}
    \label{tab:table_words}
\end{table}

\subsubsection{Título III}
El resto della concluían sayo de velarte, calzas de velludo para las fiestas, con sus pantuflos de lo mesmo, y los días de entresemana se honraba con su vellorí de lo más fino.
\paragraph{Título IV}
Tenía en su casa una ama que pasaba de los cuarenta, y una sobrina que no llegaba a los veinte, y un mozo de campo y plaza, que así ensillaba el rocín como tomaba la podadera.
\myparagraph{Título IV ii}
Frisaba la edad de nuestro hidalgo con los cincuenta años; era de complexión recia, seco de carnes, enjuto de rostro, gran madrugador y amigo de la caza.
\subparagraph{Título V}
Quieren decir que tenía el sobrenombre de Quijada, o Quesada, que en esto hay alguna diferencia en los autores que deste caso escriben; aunque por conjeturas verosímiles se deja entender que se llamaba Quijana.